\documentclass[usenames,dvipsnames]{beamer}
\usepackage{comment}
\usetheme{CambridgeUS}
%\usepackage{macros_gpi}

\usepackage{epsfig}
\usepackage[english]{babel}
\usepackage[utf8]{inputenc}
\usepackage[T1]{fontenc}

\usepackage{appendixnumberbeamer}
\usepackage{epsfig}
\usepackage[english]{babel}
\usepackage[utf8]{inputenc}
\usepackage[T1]{fontenc}
\usepackage{algorithm}
\usepackage{stmaryrd}
%\usepackage{slashbox}
\usepackage{algorithmic}
\usepackage{multirow}   
\usepackage{pstricks}    
\usepackage{color}   
\usepackage{pifont}
\usepackage{supertabular}   
\usepackage{graphicx}
\usepackage{graphbox}
\usepackage{caption}
\usepackage{subcaption}
\usepackage{animate}
%\usepackage{pdfpc-commands}
%\usepackage{xmpmulti}
\captionsetup[figure]{labelformat=empty}
\usepackage{tikz}
\usetikzlibrary{shadows}
\usepackage{fontawesome}
%\newlength{\myheight}

\usepackage{xcolor}
\definecolor{orange-perp}{rgb}{1.0,0.412,0}
\definecolor{prune-saclay}{rgb}{0.388,0,0.235}
\definecolor{bleu-nice}{rgb}{0,0.686,0.843}
\setbeamercolor{author in head/foot}{bg=black,fg=white}
\setbeamercolor{title in head/foot}{fg=black,bg=white}
\setbeamercolor{frametitle}{fg=black,bg=white}
\setbeamercolor{date in head/foot}{bg=black,fg=white}
\setbeamercolor{section in head/foot}{bg=black,fg=white}
\setbeamercolor{subsection in head/foot}{fg=black,bg=white}
\definecolor{darkspringgreen}{rgb}{0.09, 0.45, 0.27}

\setbeamercolor{block body alerted}{bg=white,fg=gray}
\setbeamercolor{block title alerted}{bg=gray,fg=white}

\setbeamercolor{block body}{bg=black!0.2,fg=black}
\setbeamercolor{block title}{bg=black,fg=white}
%\setbeamercolor{block body}{bg=structure!10}
%\setbeamercolor{block title}{bg=structure!20}

\selectlanguage{french}
\setbeamercolor{footlinecolor}{fg=white,bg=black}
\renewcommand\footnoterule{{\color{black}\hrule height 0.5pt width \paperwidth}}

\DeclareMathOperator*{\argmin}{arg\,min\ }


\usepackage{hyperref}
\hypersetup{
  %colorlinks   = true, %Colours links instead of ugly boxes
  %urlcolor     = blue, %Colour for external hyperlinks
  %linkcolor    = blue, %Colour of internal links
  %citecolor   = red %Colour of citations
}

%\setbeamercolor{itemize item}{fg=prune}
%\setbeamercolor{itemize subitem}{fg=prune}
%\setbeamercolor{itemize subsubitem}{fg=prune}

%\setbeamertemplate{itemize item}[ball]{fg=prune}
%\setbeamertemplate{itemize subitem}[ball]{fg=prune}
%\setbeamertemplate{itemize subsubitem}[triangle]

\setbeamercolor{item projected}{fg=white,bg=black}

\setbeamertemplate{itemize item}{%
    \begin{tikzpicture}
        \shade[ball color=black!100!white] (0,0) circle (0.6ex);
    \end{tikzpicture}
}

\setbeamertemplate{itemize subitem}{%
    \begin{tikzpicture}
        \shade[ball color=black!100!white] (0,0) circle (0.6ex);
    \end{tikzpicture}
}


\setbeamercolor*{title}{use=structure,fg=white,bg=black!95}
\setbeamertemplate{title page}[default][colsep=-4bp,rounded=true,shadow=true]
\setbeamertemplate{navigation symbols}{} 
%\setbeamerfont{section number projected}{size=\footnotesize}
%\setbeamercolor{section number projected}{bg=prune,fg=white}
%\setbeamercolor{section in toc}{fg=prune}
%\setbeamercolor{subsection in toc}{fg=prune}
%\setbeamercolor{subsection number projected}{bg=prune}

\setbeamercolor{section number projected}{bg=black,fg=white}
\setbeamercolor{section in toc}{fg=black}
\setbeamercolor{subsection in toc}{fg=black}
\setbeamercolor{subsection number projected}{bg=black}

%\setbeamertemplate{subsections in toc}[square]
\mode<all>

%\setbeamertemplate{footline}{
%\begin{picture}(0,0)(0,0)
%\put(320,4){\footnotesize \insertframenumber{}/\inserttotalframenumber{}}
%\end{picture}
%}
\usepackage{hyperref}
\usepackage{siunitx}

\usepackage{pdfpages}

%\includegraphics[width=1cm]{logo-projet-finance-par-ANR.jpg}

%\title[SKA computing]{\textbf{Le radiotélescope SKA}} 
%\subtitle{Un problème inverse de grande dimension à résoudre en temps réel}

\title[Problème inverse SKA]{\textbf{Problèmes inverses de grande dimension à résoudre en temps réel}} 
\subtitle{Le radiotélescope SKA}

\institute[L2S]{}%Laboratoire des Signaux et Systèmes (L2S) -  Groupe Problèmes Inverses (GPI) }
\author[  F. Orieux $\&$ N. Gac]{François  Orieux $\&$ Nicolas Gac \\ \vspace{0.5cm} \includegraphics[height=1.7 cm]{logo-saclay-white.png} \hspace{2cm}
\includegraphics[height=1.7 cm]{L2S_tutelles_vertical.pdf} }
\date[Café sciences CS]{\tiny{\textit{Café sciences CentraleSupélec, 6 avril 2023}} }



\begin{document}


%\includepdf[pages=-]{Slide attente}


\frame[plain]{\titlepage}

%\frame[noframenumbering]{\tableofcontents}
\begin{comment}
%\includegraphics[width=1cm]{logo-projet-finance-par-ANR.jpg}
\title[\textbf{Dark-era project (2021-25)}]{\textbf{Dark-era project}} 
\subtitle{Dataflow Algorithm aRchitecture co-design of SKA pipeline for Exascale Radio Astronomy\\ }
\institute[]{Daniel Charlet$^{**5}$ (IJCLab), Karol Desnos$^{1}$, Mickael Dardaillon$^{3}$, André Ferrari$^{4}$, Chiara Ferrari$^{4}$, Nicolas Gac$^{3}$, Adrien Gougeon$^{2}$, Jean-François Nezan$^{1}$, Nicolas Monnier$^{3}$, François Orieux$^{3}$, Simon Prunet$^{4}$, Martin Quinson$^{2}$, Ophélie Renaud$^{1}$, Frédéric Suter$^{**2}$(IN2P3 Computing Center), Cyril Tasse$^{**5}$ (GEPI), Cédric Viou$^{5}$, Sunrise Wang$^{3\&4}$}
\author[IETR/IRISA/L2S/Lagrange/Nançay]{$^{1}$IETR (INSA),  $^{2}$IRISA (ENS), $^{3}$L2S (CS), $^{4}$Lagrange (UCA),  $^{5}$Nançay (Obs Paris)}
\date[Café sciences CS]{\includegraphics[width=0.25\textwidth]{DARKERA_logo_color.pdf} \hfill
     { \scalebox{0.75}{\tiny{ANR-20-CE46-0001-01} \includegraphics[width=0.1\textwidth]{anr-light.pdf}  }} }
\end{comment}

\section{Problèmes inverses}

\frame{
\frametitle{Instruments et mesures complexes}

\begin{itemize}
    \item Mesures de plus en plus volumineuses et complexes
    \item Liés aux développements des instruments
    \begin{itemize}
        \item Tomographie, microscopie, \dots
        \item James Webb Space Telescope
        \item \emph{SKA}
    \end{itemize} \vfill
    \item \emph{Algorithmes} de plus en plus complexes
    \begin{itemize}
        \item Estimation d'hyper-paramètres
        \item Quantification d'incertitudes
        \item Modèles complexes (modèles instrument, IA, \dots)
    \end{itemize} \vfill
    \item \emph{Architectures} complexes d'accélération des calculs  \begin{itemize}
        \item GPU : Processeurs massivement parallèles 
        \item FPGA : Architectures dédiées
    \end{itemize} \vfill
\end{itemize}
}

\subsection{L2S}
\begin{frame}
  \frametitle{Laboratoire des Signaux et Systèmes (L2S)}

  \vfill %                                                                                                                            
  Université Paris-Saclay -- CNRS -- CentraleSupélec \vfill
  \vfill
  \begin{block}{Expertise}
      \begin{itemize}
      \item Trois pôles \(\rightarrow\) Signaux et statistiques \(\rightarrow\) Groupe Problèmes Inverses (GPI)
      \item \(\approx 30\) perm. pos. and \(\approx30\) PhD. et Postdoc
      \end{itemize}
  \end{block}
  \vfill
  \begin{block}{Applications}
    \begin{itemize}
    \item Tomographie, Contrôle Non Destructif (CND)
    \item Restauration et reconstruction d'image, super-résolution
    \item Astronomie, microscopie, imagerie industrielle
    \end{itemize}
  \end{block}
  \vfill
  \begin{block}{Méthodes}
    \begin{itemize}
    \item Approches Bayésiennes, Ondelettes, optimization, MCMC, ML
    \item Adéquation Algorithme Architecture
    \end{itemize}
  \end{block}
\end{frame}

\subsection{Le radiotelescope SKA}
\frame{
\frametitle{SKA et la radioastronomie}
\begin{center}
    \includegraphics[width=0.75\linewidth]{arecibo.png}
\end{center}
\begin{itemize}
    \item Observation de corps noir, raie H1 à 21cm, \dots
    \item Sensibilité liée à la taille
    \item Résolution liée à la taille et la longueur d'onde
\end{itemize}
}

\subsection{Le radiotelescope SKA}

\frame{
\frametitle{SKA et l'interférométrie}
\begin{center}
    \includegraphics[width=0.4\linewidth]{reseau antenne.png}
    \includegraphics[width=0.4\linewidth]{lowfar.png}
\end{center}
\begin{itemize}
    \item Augmentation du «diamètre» par interférométrie
    \item \textit{Beamforming} avec les réseaux de phase
\end{itemize}
}

\subsection{\textit{Computational imaging} pour SKA}

\frame{
\small{
\begin{columns}[t]
    \begin{column}{.4\linewidth}
\begin{center}
\includegraphics[width=\textwidth]{SKA-at-Night-768x384.jpg} \\
\vspace{0.5cm}
\includegraphics[width=0.7\textwidth]{image002}
\end{center}
\end{column}
\begin{column}{.6\linewidth}
    \begin{block}{Plus grand radiotélescope jamais construit}
          \begin{itemize}
            \item \textcolor{blue}{200+} paraboles en Afrique du Sud
            \item \textcolor{blue}{130 000+} antennes en Australie
            \end{itemize}
        \end{block}

            \begin{block}{Un pipeline fait de trois étages}
            \begin{itemize}
                \item[CSP] Le flux des antennes est corrélé pour produire des \textbf{visibilités}, c-à-d des points de Fourier.
                \item[SDP] Une image \textbf{hyperspectrale} est reconstruite à partir du plan de Fourier
                \item[SRC] Un post-traitement dans des centres régionaux
                \end{itemize}
    \end{block}
   
\end{column}
\end{columns}
}
}


\subsection{Algorithmes d'inversion}


\frame{
\centering
\includegraphics[width=0.95\linewidth]{2018-10-23_MeerKAT_Galactic-Centre_Low-Res.jpg}
}
\frame{
\frametitle{Un problème inverse}

   \begin{center}
        \begin{tikzpicture}[thick,>=stealth,scale=0.75]
           
            \draw (0,0) node[left] {$x$};
           
            \draw[->] (0,0) -- (1.3,0) coordinate (t);
           
            \draw (t) +(0,-0.35) rectangle +(1.75,0.35) +(1.75,0) coordinate (t)
            +(0.875,0) node {$H$};
           
            \draw[->] (t) -- ++(1.3,0) coordinate (t);

            \draw (t) ++(0.16,0) circle (0.16cm);

            \draw[thin] (t) +(0.07,0) -- +(0.25,0);

            \draw[thin] (t) +(0.16,0.09) -- +(0.16,-0.09);
           
            \draw[->] (t) ++(0.16,1.16) node[above] {$b$}-- ++(0,-1);
            
            \draw[->] (t) ++(0.32,0) -- ++(1.3,0) node[right] {$y$};

        \end{tikzpicture}
    \end{center}
    \vfill
    \begin{center}
        $\rightarrow$ \textit{Estimer $x$ à partir de $y$}
    \end{center}
    \vfill

    \begin{itemize}
    \item $x$ : inconnue de grande taille, l'image hyperspectrale
    \item $y$ : {\color{gray} mesures de grande taille}, bruit $b$
    \item $H$ modèle de données ou d'observation (ou instrument)
    \item Données incomplètes, dégradées, insuffisantes, \dots
    \item Inversion naïve peu satisfaisante 
    \item Algorithmes existants (CLEAN) \textit{ad-hoc} \vfill
    \end{itemize}

}

\frame{
\frametitle{Problèmes inverses mal-posés}
\centering
\vfill

\hfill{}\includegraphics[width=0.3\textwidth]{uv-cov.png}\hfill{}\includegraphics[width=0.3\textwidth]{dirty.png}\hfill{}

\vfill
\vfill\includegraphics[width=0.8\linewidth]{clean.png}
}

\frame{
\frametitle{Inversion : thèse N. Monnier (soutenance juin 2023)}
    \begin{equation*}
        \hat{x} = \argmin_{x} \| y - Hx\|^2 + \lambda \sum_c \phi_c(d_c^t x)
    \end{equation*}
    \vfill
      \begin{itemize}

  \item $\phi(.) = \ell_2$, $\phi(.) = \ell_2\ell_1$, $\phi(.) = |.|$, $\ell_0$, ML (RED, Plug \& Play, unrolling, GAN, feedforward), $d_c$ ?
\vfill
\item \textbf{Calcul du gradient}
        \begin{equation}
        \label{eq:grad}
\nabla J = H^t (H x - y) = H^t H x - H^t y            
        \end{equation}
        \item Première contribution : Eq.~\ref{eq:grad} $\rightarrow$ optimisation de la \textit{major loop}
        \vfill
        \item Perspectives : \textit{minor loop}, régularisation, \dots
  \end{itemize}
}


\section{Projet Dark-era}


\subsection{PRC ANR (2021-2025)}
\frame{
%\frametitle{Projet ANR Dark-era}
\begin{block}{Projet ANR Dark-era}
\begin{columns}
\begin{column}{0.5\textwidth}
\includegraphics[width=0.8\textwidth]{DARKERA_logo_color.pdf} 
\end{column}
\begin{column}{0.5\textwidth}
\textit{Dataflow Algorithm aRchitecture co-design of SKA pipeline for Exascale Radio Astronomy}\\
\vspace{0.2cm}
\centering
     { \scalebox{0.75}{\tiny{ANR-20-CE46-0001-01} \includegraphics[width=0.4\textwidth]{anr-light.pdf}  }}   
     \end{column}
\end{columns}
\end{block}

\begin{block}{Consortium}
\begin{itemize}
\item \textbf{L2S} \textit{- CentraleSupélec}
\item \textbf{IETR} \textit{- INSA Rennes}
\item \textbf{IRISA} \textit{- ENS Rennes}
\item \textbf{Lagrange} \textit{- Université de la Côte d'Azur}
\item \textbf{Observatoire de Nançay} \textit{- Observatoire de Paris}
\end{itemize}
\end{block}

}

\subsection{\textit{SKA computing}, un défi HPC}
\frame{
  %  \frametitle{}
%\scriptsize{

    \begin{block}{Le supercalculateur SDP}
        \begin{itemize}
            \item Très forte \textbf{puissance de calcul} \\ \textit{$\sim 125$ PFLOPS crête (efficacité visée de $\sim$10 $\%$)}   
            \item Traitement en \textbf{temps réel} du flux de données \\\textit{entrée $\sim 8$ Tb/s $\implies$ sortie $\sim 100$ Gb/s}
            \item Budget \textbf{énergétique} limité\\
            \textit{Puissance instantanée $\sim 1 MW$ pour chaque SDP} 
            \item Fossé entre les \textbf{modèles de programmation} utilisés par les astronomes et les développeurs HPC \textit{Python vs MPI/CUDA/...}
        \end{itemize}
       % $\implies$ Challenge to design a power-efficient supercomputer using low power coprocessors for the intensive computing
    \end{block}
    \begin{block}{Défi d'un co-design \textbf{logiciel/matériel}}
       % Need for time and energy performance assessments 
        \begin{itemize}
            \item Chaîne algorithmique en \textbf{flot de donnée} complexe et évolutive
            \item Supercalculateur hétérogène (CPU+GPU) à grande échelle (>100 noeuds) non encore construit
        \end{itemize}
        
        $\implies$ Nécessité d'outils de prototypage rapide pour une \textbf{estimation précoce} des performances en \textbf{temps} et en \textbf{consommation énergétique}
    \end{block}
   
%}
}


\subsection{Objectifs}
\frame{
  %  \frametitle{}
%\scriptsize{
   
    \begin{block}{Objectifs du projet Dark-era}
        \begin{enumerate}
            \item Construire \textbf{SimSDP, un outil de prototypage rapide} fournissant des simulations \textit{exascale} à partir d'une description en flot de données des algorithmes.
            \item Explorer des \textbf{accélérateurs à basse consommation} alternatifs aux GPUs : FPGA, Kalray MPPA ...
            \item Contribuer au défi calculatoire de SKA
\end{enumerate}
\end{block}
\vspace{0.2cm}
%\centering
    %\frametitle{An interdisciplinary Team}
    %\begin{block}{SimSDP}<2>
        \centering
    \includegraphics<2>[width=0.7\textwidth]{simsdp_architecture-light.png}  
%\end{block}

}


%The \textbf{SDP supercomputer} will be based on a standard HPC system combined with FPGA or application-specific architectures like GPU or the manycore Kalray Massively Parallel Processor Array (MPPA). One crucial challenge is to assess the performance both in time and energy of new complex scientific \textbf{dataflow algorithms} on not-yet-existing complex computing infrastructures. It will be hardly possible without efficient \textbf{co-design methods} and \textbf{rapid prototyping tools}.




\subsection{Une équipe interdisciplinaire}


\frame{
    %\centering
    %\vspace{0.1cm}
    %\includegraphics[width=0.2\textwidth]{DARKERA_logo_color_L}  \\
    %\centering
    %\vspace{-0.7cm}
    %\frametitle{An interdisciplinary Team}
    \includegraphics[width=\textwidth]{team_Dark-era.pdf} \\
    
\footnotesize{\textit{Prototypage rapide d’un supercalculateur dédié à la radioastronomie, L’Interdisciplinarité}. \textbf{Voyages au-delà des disciplines, CNRS Edition, 2023}}
}


%Concluion à l'oral sans slide ?
%\subsection{Conclusion}
%\frame{
%\frametitle{Travaux en cours}
%}




\setbeamercolor{background canvas}{bg=black}
\section*{C'est fini !}
\frame[t,noframenumbering]
{
  
%\frametitle{Projet ANR DARK-ERA (2/2)}
\vspace{1.1cm}
\centering
{\huge\textcolor{white}{Merci de votre attention }\\
\vspace{1.5cm}
\textcolor{white}{\small{\underline{Contacts}}}\\
\flushleft
\small{\textcolor{white}{nicolas.gac@l2s.centralesupelec.fr}
\hfill
\textcolor{white}{francois.orieux@l2s.centralesupelec.fr}}\\
\vspace{1cm}
\textcolor{white}{\small{\underline{Site web Dark-era}} : \href{https://dark-era.pages.centralesupelec.fr}{\textcolor{white}{https://dark-era.pages.centralesupelec.fr}}}}\\
%\vspace{0.5cm}
\textcolor{white}{\small{\underline{Fête de la science} : \href{https://www.youtube.com/watch?v=ro2mqx5QQnI&feature=youtu.be}{\tiny{\textcolor{white}{https://www.youtube.com/watch?v=ro2mqx5QQnI\&feature=youtu.be}}}}}\\

}



\appendix
\section*{Post traitement}
\setbeamercolor{background canvas}{bg=white}
\frame{
\begin{block}{Séparation de source pour le démélange de gaz interstellaires}
      \begin{minipage}[htb]{0.4\linewidth}
      \centering
  \includegraphics[height=4cm]{hypercube}
  \end{minipage}
  \begin{minipage}[htb]{0.59\linewidth}
      \centering
      \includegraphics[height=4cm]{courbes.png}\\
  \end{minipage}\\
  \vspace{0.5cm}
  \textit{Projet CNRS HyperStars avec CEA-Irfu/LIP6 (M.A. Miville-Deschênes)} 
  \end{block}
}
\end{document}

